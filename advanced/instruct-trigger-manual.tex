\documentclass{article}
\author{Sahar Memon, Andrew Zimolzak, Li Wei\\(VA Houston / Baylor
  College of Medicine)}
\title{InSTRuCt Study E-Trigger Manual}
\date{September, 2019}
\usepackage{listings}
%\usepackage{indentfirst}
\begin{document}
\maketitle

\section{Overview}

This manual outlines procedures for VA facilities participating in the
InSTRuCt project and enables their data-analysts/informatics personnel
to run electronic triggers (e-triggers) on the data warehouse. The SQL
code for the InSTRuCt study comprises e-triggers to identify missed
opportunities in follow-up of `red flag' findings suspicious for
several cancers, including bladder, breast, colorectal (FOBT and IDA),
hepatocellular, and lung. In general, each SQL file proceeds by--

\begin{enumerate}
\item defining the red flags that warrant additional evaluation for
  cancer (often labs or imaging),

\item excluding other explanations for the red flags, such as already
  diagnosed colon cancer, or known cause of bleeding in the upper GI
  tract (often based on ICD/CPT codes),

\item excluding patients for whom follow-up is not deemed necessary,
  and

\item excluding patients for whom appropriate follow-up was already
  done (usually based on stop codes and procedure codes).
\end{enumerate}

As an example, the e-trigger for colorectal cancer identifies patients
with positive fecal blood tests, and then excludes patients with any
of the following: advanced age, deceased status, known colon cancer,
prior colectomy, terminal illnesses or hospice care, presence of a
known diagnosis that would cause bleeding in the upper GI tract rather
than lower GI tract, and appropriate colonoscopy or GI referral.

\section{How to Apply e-Trigger Process at Your Facility}

\begin{enumerate}

\item Start your operational access to the data warehouse via your
  usual method (e.g.\ desktop or Citrix connection to SQL Server
  Management Studio software). Login to a SQL server
  (e.g.\ \texttt{vhacdwa01.vha.med.va.gov}) and authenticate (using
  either username \texttt{vha01\textbackslash{}vhabhs\ldots{}}\ plus
  password, or using Windows authentication).

\item Most changes to adapt the code to the operational database and
  schema names have been performed automatically (marked by comments
  reading ``\texttt{--altered}''). However, you will need to make
  further one-time edits to make the code function in your site's
  local database environment. The SQL code as provided uses
  placeholder names of \texttt{MyDB} and \texttt{MySchema} schema for
  outputs. Replace these with any database/schema combination where
  you have write permissions.

\item For the final project, you will be limiting queries to
  approximately one month of data at a time, and you will be running
  only 1--3 triggers depending on your facility's choice. Lines of
  code contain variable assignment statements that allow you to
  customize the query date ranges to your local station, such as:

\begin{verbatim}
set @VISN=12
--set @Sta3n=580 -- -1 all sta3n
set @run_date=getdate()
set @sp_start='2012-01-01 00:00:00'
set @sp_end='2012-12-31 23:59:59'
\end{verbatim}

\item Recommended: Run sections of the SQL file sequentially (for
  example, lines 1--132 of \texttt{Fobt.sql} cover the first two
  \texttt{INSERT INTO} operations concerning tables that were newly
  created), inspecting for errors. Alternatively: run query all at
  once, inspecting for errors. Inspection for errors should include
  not only SQL errors, but data validation through a few record
  reviews performed by the site champion. (Randomly select a few
  patients with positive e-trigger, and securely transmit last name
  and last 4 of SSN from these patients to the site champion, who will
  validate via CPRS that the sample patients have a positive red flag
  inside the time period of interest.)

\item ONLY when your site enters the ``action'' phase: translate output
  tables into reports as requested by participants in the InSTRuCt
  virtual breakthrough series (e.g.\ leadership, laboratory managers).

\end{enumerate}




\section{Estimated Analyst Time Commitment}

\subsection{Initial set up (one-time tasks)}

\textbf{For Pioneer Site: Global modifications to adapt e-triggers to
  the operational Data Warehouse: 80 person-hours}

The e-triggers were developed on the Corporate Data Warehouse using a
research database and schema. Even though research data and
operational data are both hosted on the CDW, their data schemas are
not identical. Changes to the SQL code we have developed, although not
major, are needed to make the code capable of running on the
operational schema. Since we do not have access to operational data,
we cannot confirm that adapting the code to the operational schema
will work as expected. We need to find an analyst who has operational
access to help with it. This is a one-time task that will be performed
by \textbf{one site only, not by all sites.} Once this is complete,
the updated SQL codes can be used directly by \emph{all} sites. We
estimated 80 hours of effort to make code changes to all 5 triggers.
(Actual time will depend on the extent of changes needed and the
analyst's background.)

\noindent\textbf{All sites: Local modifications to the SQL code: 40
  person-hours}

With runnable triggers, each site might need to make changes to the
SQL code which are specific to their facility. The typical changes
would be on the data server name, database name, and the
interpretations of the test/image values. For example, for an abnormal
result, some sites mark it as `A', some sites mark it as `+', some
sites use descriptions as 'many', `TNTC', or ranges such as
`\textgreater{}100', `50-2000', etc. As another example, some sites
may use a `+' to mark a normal and `++' to mark an abnormal. We
recommend that the site analyst work with local clinical groups to
best customize the test/image interpretations. We predict that the
analyst and a clinician will need to get together to review, parse,
and categorize some lab/test results. Another place that might need
attention is the lab test and procedure codes (CPT, ICDProc, LOINC
etc\ldots{}). They are standard codes, but if your site uses them in
different flavors you might need do some customization. Our estimated
time dedicated to this task will be 40 person-hours of analyst time.
The study team (VA Houston) will also provide guidance to the
pioneer site in this process.

\subsection{On-going tasks for the duration of the site's participation}

\noindent\textbf{Run SQL code to retrieve data: 4 computer hours per
  trigger per month}

The time between pressing the `Run' button to retrieve data output (if
run monthly on a single site) is few hours if no errors are
encountered. This also depends on the size of the site, network
speed/traffic, data servers' performance, and their concurrent load.
The trigger run can be counted as computer hours since the analyst
will be free to do other work once they hit the `Run' button. We have
estimated that each trigger will take 4 hours to complete in a monthly
run. Here is the breakdown based on 6 months: 4 hours x 6
months x 5 triggers = 120 computer hours.

\noindent\textbf{Ongoing modifications to SQL code: 40-person hours
  per year}

The CDW releases patch updates periodically, and this might require
ongoing minor changes/updates to SQL code, by each site analyst.
Standard codes (CPT, ICD, ICDProc, LOINC, Stopcode, etc.) tend to
change every year, with addition of new codes and removal of old
codes. These changes require corresponding updates in the SQL code.
Important note here is that you only add new codes to the SQL, do NOT
remove the old ones (this is so the e-trigger continues to capture
usage of both the historical and new codes).

\subsection{Summary}

40 person-hours one-time effort.\\
40 person-hours per year ongoing effort.\\
(Optional 80 person-hours up-front effort, which \textbf{only one} of
the approximately 10 sites needs to devote.)\\
Total computer time: 120 hours during 6 month intervention.




\section{Further reading}

\subsection{InSTRuCt project documents}

\emph{Reducing Missed Test Results Change Package}\\
\emph{InSTRuCt Study FAQ}, also known as \texttt{IIR FAQ 5.6.19.docx}\\
\emph{Standard Operating Procedure}, also known as \texttt{SOP for
  INSTRUCT\_4.5.19.pdf}\\
\emph{InSTRuCt Project Overview}, also known as \texttt{IIR Project
  Charter\_7.12.19.pdf}

\subsection{SQL file descriptions as of December 2018 (numbers are approximate)}

Fobt.sql: 2901 lines, 16 input tables, 69 output tables.  \\
HCC.sql: 3286 lines, 25 input tables, 86 output tables.  \\
IDA.sql: 3151 lines, 21 input tables, 79 output tables.  \\
Lung.sql: 3747 lines, 27 input tables, 75 output tables.  \\

\subsection{List of auxiliary files documenting the SQL code, November 2018}

\begin{verbatim}
Bladder Cancer Trigger Criteria.docx
Bladder cancer trigger- ICD 9 to ICD 10 codes.docx
Breast Cancer Trigger Criteria.docx
Breast cancer- ICD 9 to ICD 10 codes.docx
Colorectal Cancer Trigger Criteria.docx
Colorectal Cancer- ICD 9 to ICD 10 codes.docx
Hepatocellular Carcinoma Trigger Criteria.doc
Hepatocellular Carcinoma- ICD 9 to ICD 10 codes.docx
Lung Cancer Trigger Criteria.doc
Lung cancer Trigger- ICD 9 to ICD 10 codes.docx
bladder cancer ICD andCPT codes.docx
\end{verbatim}

\end{document}
